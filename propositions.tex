\documentclass[10pt]{article}
\usepackage[a5paper]{geometry}
\usepackage{mathpazo}
\renewcommand{\familydefault}{\sfdefault}

\begin{document}
	\thispagestyle{empty}
	
	\Large
	Propositions
	
	\small
	
	\begin{enumerate}
		
		\item Depending on how samples are prepared and analyzed, next-generation sequencing is suitable for detection of both base-level variants and structural variants. \textsl{(this thesis)}
		
		\item High coverage next-generation sequencing data is suitable for single-exon copy number variation detection. \textsl{(this thesis)}
		
		\item Before biological variability can be detected in next-generation sequencing, first laboratory induced variability has to be minimalized. \textsl{(this thesis)}
		
		\item International screening program criteria  are currently not fully met for opportunistic genetic screening.\textsl{(this thesis)}
		
		\item In non-invasive prenatal testing, the use of multiple independent models increases the reliability of  the prediction of presence of a trisomy from a single data set. \textsl{(this thesis)}
		
		\item The same measurement outcome in non-invasive prenatal testing gives different results for women with different prior risks of carrying a child with a trisomy. \textsl{(this thesis)}
		
		\item Noise is everything that, from a certain perspective, blocks the path between reality and measurement outcome. \textsl{(this thesis)}
		
		\item Data can be of high and low quality at the same time (depending on what information should be retrieved from the data). \textsl{(this thesis)}
		
		\item Understanding how or why is seldom as useful as understanding that things are. \textsl{(Robin Hobb, Fool's Assassin)}
		
		\item It’s not what you look at that matters, it’s what you see. \textsl{(Henry David Thoreau)}
		
	\end{enumerate}
	
	\noindent
	Propositions belonging to the doctoral thesis 'Looking through the noise: improved algorithms for genetic variant detection', by Leonard Fredericus Johansson, 2019
	
\end{document}
