\begin{appendices}
	
	\chapter{Summary}
	An important goal of genome diagnostics is to detect genetic variants that can explain the presence of a disorder in a patient or in a family. This thesis describes the development and validation of various new tools and algorithms for DNA variant detection using next-generation sequencing (NGS) data.
	
	The introduction, \textbf{chapter 1}, sets the stage by presenting the different types of genetic variants we seek to detect and the techniques available to detect them.
	
	The first part of the thesis then focuses on the detection of germline variants. \textbf{Chapter 2} describes the development of an NGS-based gene panel that targets all exons of these genes and its validation as a stand-alone diagnostic test for the detection of single nucleotide variants (SNVs) and insertions or deletions of one or a few nucleotides (indels). Earlier methods of detecting these types of variants could only sequence a short stretch of DNA in a single experiment. The NGS gene panel introduced in chapter 2 simultaneously sequenced all the then-known genes for the condition of interest (cardiomyopathy). This increased the diagnostic yield from 15\% to 40\%.
	 
	\textbf{Chapter 3} presents CoNVaDING, a tool to detect copy number variants (CNVs) – the loss or gain of a long stretch of DNA – in NGS gene panels. By selecting control samples that have similar laboratory bias to that found in the sample of interest, CoNVaDING enables detection of CNVs as small as a single exon, further increasing the possible diagnostic yield. This is demonstrated in \textbf{chapter 4}, where all three variant types (SNVs, indels and CNVs) were analyzed in a set of genes related to hereditary cancer in 2,090 anonymized patients referred for hereditary cancer diagnostics and counseling. For 211 patients (10.1\% of total), variants were detected that helped explain the cancer types that had been the reason for genetic testing, including 13 (0.6\%) because of a CNV. In addition, we aimed to determine what would be found if opportunistic screening was offered to these patients. In opportunistic screening, data produced in the context of diagnostics is also used for screening purposes. In our study all patients were screened anonymously for pathogenic and likely pathogenic variants in all tumor syndrome genes present in the panel, thus deliberately hunting for secondary findings. In addition, we searched two anonymized Dutch population cohorts for (likely) pathogenic DNA variants in the same set of genes. In this population \textgreater3\% of the individuals carry these kinds of genetic variants. However, the opportunistic screening program we examined did not yet meet the current international criteria for screening and more information is needed before opportunistic screening can be offered as an extension of diagnostic genetic testing.
	
	The second part of this thesis focuses on detection of somatic variants to provide an alternative for current diagnostic methods. These are variants that present in only some of the cells, as might be the case in cancer cells. In leukemia the presence of chromosomal variants such as translocations are used for diagnosis, prognosis and treatment decisions. \textbf{Chapter 5} describes the development of a laboratory protocol based on Targeted Locus Amplification (TLA) for a panel of 18 genes associated with acute leukemia. Using this protocol we were able to sequence DNA in the vicinity of these genes through NGS. If a translocation involving one of the 18 genes is present, DNA of the translocation chromosome that is now close to that gene will also be sequenced and identified through an analysis workflow developed for the TLA gene panel. Even though a high correlation with conventional diagnostic techniques was achieved, further development is needed to make the panel suitable for diagnostic use.
	
	The third part of this thesis describes the development of various algorithms for non-invasive prenatal testing (NIPT). Short DNA fragments are present in every pregnant woman’s blood and a subset of these are of fetal origin. NIPT uses these fragments to look for the presence of an extra chromosome in the unborn child, as is the case in Down syndrome. \textbf{Chapter 6} introduces three novel algorithms that improve NIPT data analysis by enabling more sensitive detection of the presence of the extra chromosome and providing additional quality metrics that help users to correctly interpret laboratory results. Using these algorithms trisomies can theoretically be detected in the maternal blood at a percentage of fetal DNA as low as two percent. \textbf{Chapter 7} reports on the development of \textsl{NIPTeR}, an R package that makes the new algorithms and other published algorithms for analysis of NIPT data available. To properly interpret an individual pregnant woman’s NIPT result requires taking her specific situation into account, for example her age and the characteristics of the test used. Therefore, as described in \textbf{chapter 8}, we developed the online tool \textsl{NIPTRIC} that can calculate a personalized posterior risk for women who have undergone NIPT to aid in the clinical interpretation of their test result.
	
	In the final part of this thesis I step back and reflect on the work described in the thesis. In \textbf{chapter 9} I take an epistemological stance towards NGS data analysis and discuss the justification of our beliefs regarding the knowledge obtained through NGS experiments. The goal of NGS is to determine the DNA sequence of a specific sample, but we don’t observe the DNA itself. I argue that there are sufficient grounds to state that NGS can create an adequate representation of the DNA sequence of the sample. \textbf{Chapter 10} focuses on ethical issues. Here I discuss the possible impact on society of the availability of the methods and algorithms introduced in this thesis and conclude that many of the issues that can cause ethical dilemmas have to do with uncertainty. For instance, uncertainty in knowing what will be found or uncertainty regarding whether or not a disease will develop. In \textbf{chapter 11} I discuss DNA variant detection and the different kinds of noise that can prevent detection. I posit that all DNA variant detection follows the same scheme and that in all steps, from input material to detected variant, bias and error can be introduced that can hamper the correct identification of variants and affect sensitivity and specificity. Finally, I discuss the different NGS methods that are currently available, share my perspectives on what a complete DNA sequencing procedure should look like and envision single-cell whole chromosome sequencing, with a low base call error rate and the capacity to detect base modifications.
	

	
	\chapter{Samenvatting}
	Het voornaamste doel van genoomdiagnostiek is het detecteren van genetische varianten die de aanwezigheid van een aandoening van een patiënt of binnen een familie kunnen verklaren. Dit proefschrift beschrijft de ontwikkeling en validatie van verscheidene nieuwe tools en algoritmes voor DNA-variantdetectie in next-generation sequencing (NGS) data.
	
	In \textbf{hoofdstuk 1} introduceer ik de verschillende typen genetische varianten waar we naar zoeken en de technieken die beschikbaar zijn om deze op te sporen.
	
	Het eerste deel van dit proefschrift legt de nadruk op het opsporen van erfelijke (kiemlijn) genvarianten. \textbf{Hoofdstuk 2} beschrijft de ontwikkeling van een NGS genpanel en de validatie daarvan als diagnostische test voor het vinden van single nucleotide variants (SNVs) en inserties of deleties van een of enkele nucleotiden (indels). Anders dan bij vroegere diagnostiek, waarin genen per stuk bekeken moesten worden, kunnen met genpanels veel genen gelijktijdig worden geanalyseerd. Dit maakt het mogelijk om alle genen voor een bepaalde aandoening in korte te tijd te onderzoeken. Met behulp van het ontwikkelde genpanel kon voor een groep patiënten met een mogelijke erfelijke hartspieraandoening de diagnostische opbrengst worden verhoogd van 15\% tot 40\%. 
	
	\textbf{Hoofdstuk 3} introduceert CoNVaDING. Deze tool detecteert copy number variants (CNVs) – het verlies van een lang stuk DNA of de dubbele aanwezigheid daarvan – in NGS panel data. Door controlemonsters te selecteren met een vergelijkbare systematische afwijking als het te testen monster, kan CoNVaDING CNVs detecteren die slechts een enkel exon omvatten. Het mogelijke diagnostisch rendement stijgt hierdoor nog verder. Dit wordt aangetoond in \textbf{hoofdstuk 4}, waar werd gekeken naar alle drie bovenstaande varianttypen (SNVs, indels en CNVs) in een groep genen die zijn gerelateerd aan erfelijke kanker. In het onderzoek gebruikten we data van 2,090 geanonimiseerde patiënten die verwezen waren voor diagnostiek en counseling voor verschillende typen erfelijke kanker. Van alle patiënten kon bij 211 (10.1\%) een genetische diagnose gesteld worden, inclusief 13 (0.6\%) vanwege een CNV. Hiernaast onderzochten we in deze groep patiënten wat zou worden gevonden als we hen opportunistische screening zouden aanbieden. Hiervoor werden alle patiënten in onze studie anoniem gescreend op pathogene en waarschijnlijk pathogene varianten in alle aan tumorsyndroom gerelateerde genen in het panel. Daarnaast zochten wij in twee geanonimiseerde Nederlandse populatiecohorten naar (waarschijnlijk) pathogene DNA varianten in dezelfde set genen. Meer dan 3\% van de mensen bleek drager van een dergelijke variant. Het opportunistische screeningsprogramma dat wij testten voldoet momenteel echter nog niet aan internationale screeningscriteria en meer informatie is nodig voordat opportunistische screening routinematig kan worden aangeboden.
	
	Het tweede deel van dit proefschrift richt zich op de detectie van niet-erfelijke (somatische) varianten om huidige diagnostiek te verbeteren. Somatische varianten zijn slechts in een deel van de cellen aanwezig, zoals het geval kan zijn in kankercellen. Bij leukemieën wordt de aanwezigheid van chromosomale varianten, zoals translocaties, gebruikt voor diagnose, prognose en beslissingen over de behandeling van de patiënt. \textbf{Hoofdstuk 5} beschrijft de ontwikkeling van een laboratoriumprotocol voor een panel van 18 met acute leukemie geassocieerde genen op basis van Targeted Locus Amplification (TLA). Met dit protocol kunnen we het DNA rondom deze genen sequencen met behulp van NGS. Het voordeel hiervan is dat translocaties waarbij een van de 18 genen betrokken is, kunnen worden opgespoord, ongeacht de translocatiepartner. Uit ons onderzoek bleek dat verdere ontwikkeling nodig is voordat het TLA-protocol bestaande leukemie-diagnostiek kan vervangen. 
	
	Het derde deel van dit proefschrift beschrijft de ontwikkeling van verscheidene nieuwe algoritmes voor niet-invasieve prenatale testen (NIPT). In het bloed van elke zwangere vrouw zijn korte DNA fragmenten aanwezig, die voor een deel van de foetus afkomstig zijn. NIPT gebruikt deze fragmenten om te kijken naar de aanwezigheid van een extra chromosoom bij de foetus, zoals bij Down syndroom. \textbf{Hoofdstuk 6} introduceert drie nieuwe algoritmes die NIPT data-analyse verbeteren door een gevoeliger detectie van een eventueel extra chromosoom en via extra kwaliteitsparameters voor correcte interpretatie van de laboratoriumresultaten. Met behulp van deze algoritmes kunnen, in theorie, trisomieën worden gedetecteerd in moederlijk bloed bij een percentage foetaal DNA van slechts twee procent. \textbf{Hoofdstuk 7} beschrijft de ontwikkeling van \textsl{NIPTeR}, een R package waarin we de nieuwe algoritmes en andere gepubliceerde algoritmes voor NIPT data-analyse publiekelijk beschikbaar maken. Om voor een individuele zwangere vrouw de uitslag van een NIPT correct te kunnen interpreteren, moet er rekening worden gehouden met haar specifieke situatie, zoals haar leeftijd en de karakteristieken van de gebruikte test. Daarom wordt in \textbf{hoofdstuk 8} de online tool \textsl{NIPTRIC} geïntroduceerd. Met behulp van deze tool kan een meer gepersonaliseerde, en daarmee nauwkeuriger, NIPT uitslag worden berekend ter ondersteuning van een klinische interpretatie van het onderzoeksresultaat. 
	
	In het laatste deel van dit proefschrift reflecteer ik op het werk dat in dit proefschrift wordt beschreven. In \textbf{hoofdstuk 9} kijk ik met een epistemologisch perspectief naar NGS data-analyse en bediscussieer de rechtvaardiging van onze overtuigingen met betrekking tot kennis verkregen via NGS experimenten. Het doel van NGS is om de DNA-volgorde van een specifiek monster te bepalen, maar het DNA zelf observeren we daarbij niet. Ik beargumenteer dat er voldoende grond is om te stellen dat NGS een adequate representatie kan geven van de DNA-volgorde van een monster. \textbf{Hoofdstuk 10} focust op ethische kwesties. Ik bediscussieer de mogelijke invloed op de maatschappij die de beschikbaarheid van de in dit proefschrift geïntroduceerde methoden en algoritmes hebben en concludeer dat veel ethische dilemma’s te maken hebben met onzekerheid, zoals onzekerheid over wat er zal worden gevonden of onzekerheid over de vraag of een ziekte tot ontwikkeling zal komen of niet. In \textbf{hoofdstuk 11} stel ik dat alle DNA-variantdetectie hetzelfde schema volgt en dat in alle stappen vanaf het startmateriaal tot aan de gedetecteerde variant, systematische afwijkingen en fouten kunnen worden geïntroduceerd. Dit kan de correcte identificatie van varianten belemmeren en de sensitiviteit en specificiteit van een test beperken. Tot slot bediscussieer ik verschillende beschikbare NGS methoden en geef mijn toekomstperspectief over DNA-sequencing. Ik stel me een methode voor waarbij per cel het hele genoom kan worden bepaald, met weinig fouten in het benoemen van de basen en de mogelijkheid om base-modificaties te detecteren
	
	
	
	\chapter{Acknowledgements}
	
	Het schrijven van dit dankwoord is misschien het moeilijkste deel van dit proefschrift, want in de jaren dat ik hier op de afdeling genetica werk hebben zoveel mensen mij bijgestaan, zowel voor als tijdens mijn promotietraject. Het is daarom onmogelijk om iedereen die dank verdient bij naam te noemen. Enkele mensen wil ik toch bij naam noemen.\\
	
	\noindent Allereerst, Dr. Raddatz, beste Birgit, reeds in 2001, tijdens mijn stage op de afdeling genetica was jij mijn begeleider. Samen met Gineke en de rest van de FISH-groep heb je me geïntroduceerd in de genetische diagnostiek. Na een korte uitstap bij de prenatale cytogenetische diagnostiek, kwam ik in 2005 weer bij jou in de groep te werken. Ook na de reorganisatie en de oprichting van ‘team 3’ volgde ik jou naar de nieuwe werkzaamheden. In de afgelopen (bijna) 18 jaar heb ik zoveel van je geleerd en je stond (en staat) altijd klaar voor een vraag of steuntje in de rug wanneer ik het nodig heb, of om me naar huis te sturen wanneer ik te lang doorwerk. Ik ben daarom ook blij dat je mijn co-promotor bent.\\ 
	
	\noindent Prof. Dr. Sijmons, beste Rolf, onze wegen hebben elkaar voor het eerst gekruist bij de heroprichting van de Genehoppers waar we ons met de belangrijkste activiteit van de afdeling genetica bezig hielden: muziek maken. Bij de oprichting van de sectie ontwikkeling en innovatie in 2012 waarin ‘team 3’ werd opgenomen, kwam ik bij jou werken. Ik zal nooit vergeten dat ik enkele jaren later vertelde dat ik na het afronden van mijn master wijsbegeerte (dat nog een aantal jaren in de toekomst lag) graag een PhD traject in zou willen gaan. Je reactie was: “goed idee, maar we gaan niet wachten tot je de master hebt afgerond”. Het gevolg is dat ik hier nu zit, nog steeds zonder master in de wijsbegeerte, maar met proefschrift. Hartelijk dank voor alle kansen die je me hebt geboden, de vrijheid die je me hebt gegeven en je onaflatende enthousiasme.\\ 
	
	\noindent Prof. Dr. Swertz, beste Morris, ik wil ook jou hartelijk bedanken voor de kansen die je me hebt geboden en de begeleiding in de afgelopen jaren. In 2012 gaf je me de gelegenheid om een stage te lopen binnen de GCC en mede daaruit zijn de onderwerpen ontstaan die uiteindelijk de basis vormen van dit proefschrift. Ik ben ook erg blij dat ik na mijn promotie als postdoc in jouw team aan de slag kan.\\ 
	
	\noindent I would like to express my gratitude to Prof. Dr. V.V.A.M Knoers, Prof. Dr. J.K. Ploos van Amstel and Prof. Dr. M. Vihinen for making room in your schedules to assess my thesis.\\ 
	
	\noindent Prof. dr. J.W. Romeijn, beste Jan-Willem, hartelijk dank voor het commentaar op de epistemologische reflectie.\\
	
	\noindent Ook wil ik mijn paranimfen bedanken. Ik ben vereerd dat jullie hier vandaag naast mij willen staan. Eddy, ik ken je al sinds ik bij de prenatale cytogenetica kwam werken en de bijdrage die je aan dit proefschrift hebt geleverd is van onschatbare waarde. Elk jaar kijk ik ook uit naar onze reizen. Met jou heb ik de grootste avonturen beleefd op zoek naar de grootste slagvelden van de afgelopen eeuwen. Samen hebben we de slag bij Waterloo overleefd en na ons uitstapje naar Borodino weet ik dat we vanuit elke plaats de weg naar huis terug zullen vinden. Krista, bij de oprichting van ‘team 3’ werden we collega’s. Jij met de DNA expertise en ik vanuit de chromosomenhoek. Ik heb veel van je geleerd over het detecteren en interpreteren van varianten.\\ 
	
	\noindent I would like to thank all the past and current colleagues of the O\&I section for interesting discussions and collaborations.\\ 
	
	\noindent Mohamed, we have worked closely together the past four years. I have enjoyed our discussions and the ‘African viewpoint’. It was a sad moment when you returned to Tanzania, but I look forward to come and visit you in the near future.\\
	
	\noindent Helga, thank you for your work on ‘the onco paper’ and your advice on people/promotor management.\\
	
	\noindent A warm thanks to all past and present GCC colleagues. I remember when the GCC started within the department and I had to share the coffee machine with those strange bio-informaticians. In the past years we have also shared quite a lot of beers and hotpots. Undoubtedly many more will follow. I am happy to get the opportunity to work more closely with you in the coming years.\\
	
	\noindent Freerk, toen mijn stage bij de GCC begon was jij mijn buurman en tot de introductie van de flexplekken zijn we dat gebleven. Bedankt voor de hulp bij al mijn domme vragen en voor de prettige samenwerking bij het ontwikkelen van CoNVaDING. Veel succes bij je nieuwe baan in Utrecht.\\
	
	\noindent Edgar, bedankt voor de gezelligheid. Je aanwezigheid wordt nog elke dag gemist in de ‘stilte’-kamer.\\
	
	\noindent I would like to thank Jackie Senior and Kate Mc Intyre for editing all the chapters that made it into this thesis. I am sorry for the many words I sent for the reflection/discussion.\\ 
	
	\noindent Ook bedankt aan alle stagiaires die ik de afgelopen jaren heb mogen begeleiden. Ik heb wellicht meer geleerd van jullie dan jullie van mij. Dirk, bedankt voor je onmisbare bijdrage aan het NIPT project.\\
	
	\noindent Huidige en oud-collega’s van de genoomdiagnostiek, bedankt voor de goede samenwerking en de begeleiding in mijn jonge jaren. De jaren aan ADL4 zal ik nooit vergeten.\\
	
	\noindent Dr. E.L.M. Maeckelberghe, beste Els, en alle deelnemers aan de maandelijkse ethiekbijeenkomsten, heel erg bedankt voor de discussies over de meest uiteenlopende onderwerpen.  Jullie hebben mijn perspectief aanzienlijk verbreed.\\
	
	\noindent Mama en papa, bedankt dat jullie mij een geweldige jeugd hebben gegeven en dat jullie nog altijd voor mij klaar staan. Een bezoek aan jullie is altijd een rustpunt in het drukke leven. Papa, helaas is er geen diner met Carl XVI Gustaf gepland, dat bleek toch bij een ander feestje te horen.\\
	
	\noindent Carien, Johan, Gerhard, Rianne en Leonie, bedankt dat de deur bij jullie altijd open staat. Bij jullie is het altijd gezellig.\\
	
	\noindent Lieve Kim, ik ben elke dag blij dat je er bent. Het leven is zoveel leuker met jou en de jongens. Bedankt voor het aanhoren van al mijn verzuchtingen. Alexander en Thomas, bedankt voor alle leuke dingen die ik met jullie mag doen en dat jullie mij leren dat er meer in het leven is dan werken (zoals trampoline springen). 
	
	\chapter{About the author}
	
	\begin{wrapfigure}{r}{0.33\textwidth}
		\begin{center}
			\includegraphics[width=0.25\textwidth]{img/Lennart_Johansson}
		\end{center}
	\end{wrapfigure}
	Leonard Fredericus (Lennart) Johansson was born in Uithuizen, The Netherlands, on May 29th 1980. 
	After completion of the VWO at Het Hogeland College in Warffum in 1998 he followed the biology and medical laboratory research track of the bachelor Life Sciences and Technology at the Hanze University Groningen and obtained a BASc in 2002, graduating on the implementation of comparative genomic hybridization for the purpose of hematological malignancies at the University Medical Center Groningen (UMCG) department of genetics. 
	In 2015 he obtained a BA in philosophy at the University of Groningen. 
	In his thesis he formulated an answer to the question: ‘does new knowledge about existing results give a moral duty to recontact in clinical genetics?’. 
	He has been working at the UMCG department of genetics since 2002, starting in genome diagnostics and later switching to the development and innovation section where he started his PhD in 2015 under the guidance of Rolf Sijmons, Birgit Raddatz and Morris Swertz, of which the results are presented in this thesis.
	
	\chapter{List of publications}
	
	\section*{Published}
	\begin{enumerate}
		\item Targeted next-generation sequencing can replace Sanger sequencing in clinical diagnostics. Sikkema-Raddatz B, \textbf{Johansson LF}, de Boer EN, Almomani R, Boven LG, van den Berg MP, Spaendonck-Zwarts KY, van Tintelen JP, Sijmons RH, Jongbloed JDH, Sinke RJ. \textsl{Human Mutation}. 2013;34(7):1035-42.
	\item Successful noninvasive trisomy 18 detection using single molecule sequencing. van den Oever JME, Balkassmi S, \textbf{Johansson LF}, Adama van Scheltema PN, Suijkerbuijk RF, Hoffer MJV, Sinke RJ, Bakker E, Sikkema-Raddatz B,  Boon EMJ. \textsl{Clinical Chemistry}. 2013;59(1):705-9.
	\item Next-generation sequencing-based genome diagnostics across clinical genetics centers: implementation choices and their effects. Vrijenhoek T, Kraaijeveld K, Elferink M, de Ligt J,  Kranendonk E, Santen G, Nijman IJ, Butler D, Claes G, Costessi A, Dorlijn W, van Eyndhoven W, Halley DJJ, van den Hout MCGN, van Hove S, \textbf{Johansson LF}, Jongbloed JDH, Kamps R, Kockx CEM, de Koning B, Kriek M, Lekanne dit Deprez R, Lunstroo H, Mannens M, Mook OR, Nelen M, Ploem C, Rijnen M, Saris JJ, Sinke R, Sistermans E, van Slegtenhorst M, Sleutels F, van der Stoep N, van Tienhoven M, Vermaat M, Vogel M, Waisfisz Q, Weiss JM, van den Wijngaard A, van Workum W, Ijntema H, van der Zwaag B, Van Ijcken WFJ, den Dunnen J, Veltman JA, Hennekam R, Cuppen E. \textsl{European Journal of Human Genetics}. 2015;23(9);1142-50.
	\item Whole-exome sequencing is a powerful approach for establishing the etiological diagnosis in patients with intellectual disability and microcephaly. Rump P, Jazayeri O, van Dijk-Bos KK, \textbf{Johansson LF}, van Essen AJ, Verheij JBGM, Veenstra-Knol HE, Redeker EJW, Mannens MMAM, Swertz MA, Alizadeh BZ, van Ravenswaaij-Arts CMA, Sinke RJ, Sikkema-Raddatz B. \textsl{BMC medical genomics}. 2016;9(1):7
	\item CoNVaDING: single exon variation detection in targeted NGS data. \textbf{Johansson LF}*, van Dijk F*, de Boer EN, van Dijk-Bos KK, Jongbloed JDH, van der Hout AH, Westers H, Sinke RJ, Swertz MA, Sijmons RH, Sikkema-Raddatz B. \textsl{Human Mutation}. 2016;37(5):457-64.
	\item NIPTRIC: an online tool for clinical interpretation of non-invasive prenatal testing (NIPT) results. Sikkema-Raddatz B, \textbf{Johansson LF}, de Boer EN, Boon EMJ, Suijkerbuijk RF, Bouman K, Bilardo CM, Swertz MA, Dijkstra M, van Langen IM, Sinke RJ, te Meerman GJ. \textsl{Scientific reports}. 2016;6:38659.
	\item Novel algorithms for improved sensitivity in non-invasive prenatal testing. \textbf{Johansson LF}*, de Boer EN*, de Weerd HA, van Dijk F, Elferink MG, Schuring-Blom GH, Suijkerbuijk RF, Sinke RJ, te Meerman GJ, Sijmons GJ, Swertz MA, Sikkema-Raddatz B. \textsl{Scientific reports}.2017;7;1838.
	\item Genetic Screening Test to Detect Translocations in Acute Leukemias by Use of Targeted Locus Amplification. Alimohamed MZ*, \textbf{Johansson LF}*, de Boer EN, Splinter E, Klous P, Yilmaz M, Bosga A, van Min M, Mulder AB, Vellenga E, Sinke RJ, Sijmons RH, van den Berg E, Sikkema-Raddatz B. \textsl{Clinical Chemistry}. 2018;64(7):1096-1103.
	\item NIPTeR: an R package for fast and accurate trisomy prediction in non-invasive prenatal testing. \textbf{Johansson LF}, de Weerd HA, de Boer EN, van Dijk F, te Meerman GJ, Sijmons RH, Sikkema-Raddatz B, Swertz MA. \textsl{BMC bioinformatics}. 2018;19(1):531.
	\end{enumerate}
	
	\section*{Accepted}
	\begin{enumerate}
		\item A next-generation sequencing method for gene doping detection that distinguishes low levels of plasmid DNA against a background of genomic DNA. de Boer EN, Ettema P, \textbf{Johansson LF}, van Diemen CC, Haisma HJ. \textsl{Gene Therapy}.
	\end{enumerate}
	
	\section*{Submitted}
	\begin{enumerate}
		\item What if we would use a diagnostic multi-cancer gene panel for opportunistic screening? A study in 2,090 Dutch familial cancer patients.  \textbf{Johansson LF}, van Dijk-Bos KK, van der Hout AH, Knopperts AP, Leegte B, van den Akker PC, Kok K, van Langen IM , Swertz MA, Weersma RK, Sinke RJ, Sikkema-Raddatz B, Westers H*,  Sijmons RH*.
		\item Reinterpretation, reclassification, and its downstream effects: Challenges for clinical laboratory geneticists. el Mecky J*; \textbf{Johansson LF}*; Plantinga M; Fenwick A; Lucassen A; Dijkhuizen T; van der Hout A; Lyle K; van Langen I.
		\item Variant re-interpretation in next-generation sequencing data increases diagnostic yield in 2002 Dutch cardiomyopathy patients. Alimohamed MZ, \textbf{Johansson LF}, Posafalvi A, Boven LG, van Dijk KK, Walters L, Vos YJ, Westers H, Hoedemaekers YM, Sinke RJ, Sijmons RH, Sikkema-Raddatz B, Jongbloed JDH*, van der Zwaag PA*
		\item VIBE: a pipeline-friendly software tool for genome diagnostics to prioritize genes by matching patient symptoms to literature. van der Velde KJ*, van den Hoek S*, van Dijk F, van Diemen CC, \textbf{Johansson LF}, Abbott KM, Deelen P, Sikkema-Raddatz B, Swertz MA. 
		\item Unresolved severe hypercholesterolemia in a cohort of Emiratis with high prevalence of type 2 diabetes mellitus. Rimbert A, Kuivenhoven JA, Buckley A, Lansberg P, Viel M, Kanninga R, \textbf{Johansson LF}; Dullaart RPF, Sinke RJ, al Tikriti A, Daggag H, Taysir Barakat M.
	\end{enumerate}
	
	\section*{Dutch}
	\begin{enumerate}
		\item Next-generation sequencing nader bekeken. de Boer EN, \textbf{Johansson LF}. \textsl{Analyse}. september 2014;69(4):125-9 \\
	\end{enumerate}
	
	* Shared first or last author
	
	\chapter{Other academic activities}
	
	\begin{table}
	\caption*{\textbf{Courses followed}}
		\footnotesize
		\begin{tabulary}{\linewidth}{LL}
			Date & Activity \\
			\hline
			\rule{0pt}{2.6ex}\mbox{08-10-2014 } & Followed Hanze University course 'programming in Python'  \\
			\rule{0pt}{2.6ex}\mbox{04-03-2015 } & Followed UMCG central library introduction \\
			\rule{0pt}{2.6ex}\mbox{13-03-2015 } & Followed GSMS workshop 'search strategy in Pubmed'\\
			\rule{0pt}{2.6ex}\mbox{25-06-2015 } & Followed GsMs course 'Project Management' \\
			\rule{0pt}{2.6ex}\mbox{01-09-2015 } & Followed GSMS course 'Scientific writing A-Z long track' \\
			\rule{0pt}{2.6ex}\mbox{10-09-2015 } & Followed GSMS course 'Scientific integrity'\\
			\rule{0pt}{2.6ex}\mbox{12-01-2016 } & Followed UMCG dept. Epidemiology course 'Medical Statistics'\\
			\rule{0pt}{2.6ex}\mbox{14-04-2016 } & Attended GATK Workshop, Edinburgh\\
			\rule{0pt}{2.6ex}\mbox{07-03-2016 } & Followed GSMS course 'Introduction to R'\\
			\rule{0pt}{2.6ex}\mbox{13-05-2016 } & Followed GSMS course 'Presentation skills'\\
			\rule{0pt}{2.6ex}\mbox{09-09-2016 } & Followed RuG Master course 'Scientific Representation'\\
			\rule{0pt}{2.6ex}\mbox{04-02-2017 } & Followed GSMS course 'Teacher training for PhDs'\\
			\rule{0pt}{2.6ex}\mbox{04-09-2017 } & Followed RuG Master course 'Publieksgericht schrijven'\\
			\rule{0pt}{2.6ex}\mbox{18-09-2017 } & Course 'Identifying genes for Mendelian traits using NGS data' at Max Delbrück center, Berlin\\
			\hline
		\end{tabulary}
\label{table:appendix_activities_1}
\end{table}			
			

\begin{table}
	\caption*{\textbf{Presentations}}
	\footnotesize
	\begin{tabulary}{\linewidth}{LL}
		Date & Activity \\
		\hline
			\rule{0pt}{2.6ex}\mbox{06-06-2014 } & Oral presentation at 10th Wim Schellekens symposium, The Hague \\
			\rule{0pt}{2.6ex}\mbox{02-10-2014 } & Presented poster at the NVHG conference, Papendal \\
			\rule{0pt}{2.6ex}\mbox{30-04-2015 } & Oral presentation at 13th international symposium on mutation in the genome, Leiden \\
			\rule{0pt}{2.6ex}\mbox{05-02-2016 } & Oral presentation at the BeSHG \& NVHG first joint meeting 2016, Leuven \\
			\rule{0pt}{2.6ex}\mbox{19-10-2016 } & Presented poster at the ASHG conference, Vancouver\\
			\rule{0pt}{2.6ex}\mbox{16-06-2018 } & Presented poster at ESHG conference, Milan \\
			\hline
\end{tabulary}
\label{table:appendix_activities_2}
\end{table}

\clearpage



\begin{table}
	\caption*{\textbf{Other conferences attended}}
	\footnotesize
	\begin{tabulary}{\linewidth}{p{1.1cm}L}
		Date & Activity \\
		\hline
		\rule{0pt}{2.6ex}\mbox{11-06-2015 } & Attended GSMS PhD development conference 2015 \\
		\rule{0pt}{2.6ex}\mbox{18-10-2016 } & Attended Human Genome Variation Society symposium, Vancouver \\
		\rule{0pt}{2.6ex}\mbox{31-03-2017 } & Attended the 'landelijke analistendag genoomdiagnostiek', Utrecht \\
		\rule{0pt}{2.6ex}\mbox{11-12-2017 } & Participated in the UMCG genetics department research retreat \\
		\rule{0pt}{2.6ex}\mbox{12-03-2018 } & Attended joint meeting UK/Dutch clinical genetics societies and cancer genetics groups (day 1), Nijmegen \\
		\rule{0pt}{2.6ex}\mbox{16-04-2019 } & Attended X-omics festival, Nijmegen \\
		\hline
	\end{tabulary}
	\label{table:appendix_activities_3}
\end{table}



	\begin{table}
		\caption*{\textbf{Teaching and student supervision}}
		\footnotesize
		\begin{tabulary}{\linewidth}{LL}
			Date & Activity \\
			\hline		
			\rule{0pt}{2.6ex}\mbox{01-09-2010 } & Supervized internship and graduation student Martijn Viel\\
			\rule{0pt}{2.6ex}\mbox{01-02-2011 } & Supervized internship student Jeroen Bremer \\
			\rule{0pt}{2.6ex}\mbox{01-09-2011 } & Supervized internship and graduation student Marloes Benjamins \\
			\rule{0pt}{2.6ex}\mbox{01-05-2014 } & Supervized internship and graduation student Dirk de Weerd \\
			\rule{0pt}{2.6ex}\mbox{02-06-2014 } & Guest lecture for 1st year students LS\&T Hanze Univerisity \\
			\rule{0pt}{2.6ex}\mbox{01-09-2014 } & Supervized internship and graduation student Nils Kooistra \\
			\rule{0pt}{2.6ex}\mbox{05-02-2015 } & Taught RuG course segment 'Computational molocular biology research'\\
			\rule{0pt}{2.6ex}\mbox{01-09-2015 } & Supervized internship and graduation student Roald Mulder \\
			\rule{0pt}{2.6ex}\mbox{01-02-2016 } & Taught RuG course segment 'Computational molocular biology research' \\
			\rule{0pt}{2.6ex}\mbox{05-09-2016 } & Supervized internship and graduation student Myrthe Frans\\
			\rule{0pt}{2.6ex}\mbox{06-02-2017 } & Supervized student Lars Santema for course 'Computational molocular biology research' \\
			\rule{0pt}{2.6ex}\mbox{24-05-2017 } & Guest lecture for 1st year students LS\&T Hanze Univerisity \\
			\rule{0pt}{2.6ex}\mbox{07-06-2017 } & Gave lecture in RuG course 'Human genetics and genomics' \\
			\rule{0pt}{2.6ex}\mbox{04-09-2017 } & Supervized internship and graduation student Tim suichies \\
			\rule{0pt}{2.6ex}\mbox{04-12-2017 } & Supervized internship student Tom Baars\\
			\rule{0pt}{2.6ex}\mbox{07-06-2018 } & Gave lecture in RuG course 'Human genetics and genomics' \\
			\rule{0pt}{2.6ex}\mbox{01-09-2018 } & Supervized internship and graduation student Iwan Hidding\\
			\rule{0pt}{2.6ex}\mbox{01-04-2019 } & Supervized graduation Anouk Lugtenberg \\
			\rule{0pt}{2.6ex}\mbox{05-06-2019 } & Gave lecture in RuG course 'Human genetics and genomics' \\
			\hline
		\end{tabulary}
		\label{table:appendix_activities_4}
	\end{table}
	
	\begin{table}
		\caption*{\textbf{Manuscript review}}
		\footnotesize
		\begin{tabulary}{\linewidth}{LL}
			Date & Activity \\
			\hline
			\rule{0pt}{2.6ex}\mbox{16-11-2015 } & Manuscript review for BMC Research notes (under supervision)\\
			\rule{0pt}{2.6ex}\mbox{10-03-2016 } & Manuscript review for Human Mutation (under supervision)\\
			\rule{0pt}{2.6ex}\mbox{01-11-2016 } & Manuscript review for Molecular Diagnosis \& Therapy (under supervision)\\
			\rule{0pt}{2.6ex}\mbox{10-11-2016 } & Manuscript review for International Journal of Molecular Sciences (under supervision)\\
			\rule{0pt}{2.6ex}\mbox{07-02-2017 } & Manuscript review for Welcome Open Research (under supervision)\\
			\rule{0pt}{2.6ex}\mbox{06-12-2017 } & Manuscript review for Clinical Genetics (under supervision)\\
			\rule{0pt}{2.6ex}\mbox{12-10-2018 } & Manuscript review for Welcome Open Research (under supervision)\\
			\rule{0pt}{2.6ex}\mbox{03-12-2018 } & Manuscript review for Scientific Reports (under supervision)\\
			\rule{0pt}{2.6ex}\mbox{08-02-2019 } & Manuscript review for PLOS One\\
			\rule{0pt}{2.6ex}\mbox{25-03-2019 } & Manuscript review for Atlas of Translational Medicine\\
			\rule{0pt}{2.6ex}\mbox{22-04-2019 } & Manuscript review for Bioinformatics \\
			\rule{0pt}{2.6ex}\mbox{11-06-2019 } & Manuscript review for Journal of Medicine and Life\\
			\hline
		\end{tabulary}
		\label{table:appendix_activities_5}
		\end{table}
			
			
	\begin{table}
		\caption*{\textbf{Other activities}}
		\footnotesize
		\begin{tabulary}{\linewidth}{LL}
			Date & Activity \\
			\hline	
			\rule{0pt}{2.6ex}\mbox{2014-current } & Member of VKGL NGS working group \\
			\rule{0pt}{2.6ex}\mbox{2015-current } & Member of Working field advisory committee Hanze University LS\&T bioinformatics \\
			\rule{0pt}{2.6ex}\mbox{2016-current } & Participant in monthly ethics discussion meetings chaired by Dr. E.L.M. Maeckelberghe \\
			\hline
		\end{tabulary}
		\label{table:appendix_activities_6}
	\end{table}

\begin{table}
	\caption*{Award}
	\footnotesize
	\begin{tabulary}{\linewidth}{LL}
		Date & Activity \\
		\hline
		\rule{0pt}{2.6ex}\mbox{06-06-2014} & Best junior presentation, 10th Wim Schellekens symposium, The Hague \\
		\hline
	\end{tabulary}
	\label{table:appendix_activities_7}
\end{table}

\clearpage
	
\end{appendices}

