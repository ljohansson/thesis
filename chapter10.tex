\chapter[What should I do?]{What should I do? \newline {\fontsize{18}{32}\selectfont Ethical issues in genetic analysis}}
\chaptermark{What should I do?}
\label{chap:ethics}

{ \Large \leftwatermark{
		\put(-67,-66.5){ 1 }
		\put(-67,-91.5){ 2 }
		\put(-67,-116.5){ 3 }
		\put(-67,-141.5){ 4 }
		\put(-67,-166.5){ 5 }
		\put(-67,-191.5){ 6 }
		\put(-67,-216.5){ 7 }
		\put(-67,-241.5){ 8 }
		\put(-67,-266.5){ 9 }
		\put(-76.5,-300){\includegraphics[scale=0.8]{img/thumbindex.eps}} \put(-67,-291.5){ {\color{white} 10 }}
		\put(-67,-316.5){ 11 }
		
	} \rightwatermark{
		\put(350.5,-66.5){ 1 }
		\put(350.5,-91.5){ 2 }
		\put(350.5,-116.5){ 3 }
		\put(350.5,-141.5){ 4 }
		\put(350.5,-166.5){ 5 }
		\put(350.5,-191.5){ 6 }
		\put(350.5,-216.5){ 7 }
		\put(350.5,-241.5){ 8 }
		\put(350.5,-266.5){ 9 }
		\put(346.5,-300){\includegraphics[scale=0.8]{img/thumbindex.eps}} \put(350.5,-291.5){ {\color{white} 10 }}
		\put(350.5,-316.5){ 11 }
}}


\newpage

\noindent In the previous chapter I reflected on the extent of genetic knowledge. 
We have seen that by taking different perspectives on variant detection, using various laboratory techniques, empirically sound results are obtained. 
In his book ‘Moralizing Technology’, Peter-Paul Verbeek states that designing technology comes with moral responsibility \cite{Verbeek_2011}[p.90]. 
As designers of new techniques and improved analysis methods, we cannot turn our backs on the actual and foreseeable consequences of the presence of these technologies. 

In this chapter I use Verbeek’s theory of to reflect on our novel methods and algorithms in order to better understand the possible impact they can have on society. 
Yes, we have new methods and algorithms for genetic variation detection, but do we (always) want to use them? I will not give a definitive answer to this question, and it is very likely that more questions will be asked than answered. 
This chapter should be read, instead, as a personal search for how to look at the work presented in this thesis. 
Hopefully, this section will provide a helping hand for those considering the role of technology in moral decision making within human genetics. 



\section{Moralizing technology} \label{Moralizing}
In the classical framework, belonging to a moral community is reserved for persons, most notably rational adult human beings\footnote{‘person’ does not equate with ‘people’. Who qualifies for personhood is subject of ongoing discussion \cite{White_2013}.}. 
These persons can make a distinction between what is right and what is wrong and steer their behavior accordingly – although opinions on what is right and wrong may differ. 
In current society many technologies have been introduced. Verbeek argues that technologies are not neutral instruments, but actively shape human behavior \cite{Verbeek_2011}[p.1].
To illustrate this Verbeek uses a famous example from Bruno Latour: the speedbump. 
This road construction is designed to slow down road users who apparently need a physical reminder to behave morally and not drive faster than allowed. 
If well-constructed, the speedbump even enforces this behavior. Of course, the speed bump itself is not able to distinguish right from wrong, but nevertheless it plays an important role in the relation between a person and the outside world. 
Verbeek states that ‘[a]rtifacts are morally charged; they mediate moral decisions, shape moral subjects, and play an important role in moral agency’[p.21]. 
Verbeek goes on to explain that ‘the moral significance of technology is to be found not in some form of independent agency but in the technological \textsl{mediation} of moral actions and decisions – which needs to be seen as a form of agency itself’ [p.61]\footnote{Italics from Verbeek}. 
Rational moral decision-making is still left to the person involved, but the technology (or more precise the technology in use) changes the relation between decision maker and the world, whether this shift is intended, as with the speed bump, or unintended. 
As an example of unintended moral mediation, Verbeek uses the obstetric ultrasound. 
Here, I want to include an extensive summary on this topic because it introduces various topics that are crucial in discussing the morality of genetic analysis, most notably: \textbf{changing relationships}, the \textbf{generation of situations of choice} and \textbf{intentionality of techniques}.

Obstetric ultrasound has three main effects, the first one is intended and the other two are derivative unintended consequences. 
First of all, the ultrasound visualizes, with limited resolution, the unborn child, for instance to perform an anatomical survey \cite{Platt_2013,Bahtiyar_2015}. 
During this process, especially when using 3D sonography, the expecting parents can see the fetus. Because ultrasound provides an image that stresses the human features – even if the fetus is only around 5 cm tall at 12-weeks of pregnancy – it can change the relation of expecting parents towards their unborn child. 
The perception may shift from fetus to unborn child and create a stronger bond between parents and their offspring-to-be. 
In contrast, if the sonography shows that the child has a congenital disease or syndrome, the relation may be changed in a different manner.
In this situation the parents are given a decision that they would not have without ultrasound: Do they consider the detected abnormalities to be a reason for abortion? 
Or do they choose to welcome their child into the world? Verbeek stresses that such a decision can’t be separated from the technology used. 
Verbeek writes that ‘[d]ecisions about abortion, after an ultrasound scan (and subsequent amniocentesis) have shown that the unborn child is suffering\footnote{The notion of ‘suffering’ implies that there is a subjective experience of pain, demanding cognitive capacities. It is debatable whether this notion is applicable to fetuses \cite{Derbyshire_2006}. In the context of this reflection, I equated ‘the unborn child is suffering from a serious disease’ with ‘a serious disorder has been identified in the unborn child’.} from a serious disease, are not taken autonomously by human beings – as fountainheads of morality – but in close interaction with these technologies that open up specific interpretations and actions and generate specific situations of choice’ \cite{Verbeek_2011}[p.21-22]. 
Later on in his book Verbeek takes a closer look at the exact role of ultrasound technology in the experience of a pregnancy. 
Apart from being used as a tool to represent reality, an intentionality is added to its use in how the representations are created. 
For instance ‘[m]easuring the nape of the fetus [...] is directed at detecting “defects”’.
This intentionality is added to the intentions of the expecting parents, when a decision about abortion has to be made’[p.149]. 
In this context, providing obstetric ultrasound is not only an autonomy-enhancing technique, it actively shapes the decision-making process. 
These consequences of changing the relations between parents and child, giving them new choices, or even changing the relation of a person towards him- or herself may be unintended from the perspective of the designer, but are foreseeable during design. 
Furthermore, government or social powers may be at work, expecting people to make use of technology and change their moral behavior.

These examples show that the introduction and use of new techniques can have moral consequences that deserve an assessment. 
In the following part of this chapter I will reflect on the different topics for which we have created methods and algorithms. 

\section{Moral decisions in Non-Invasive Prenatal \newline Testing}             
Since our focus is already on prenatal testing, I will start with the ethics of non-invasive prenatal testing (NIPT). 
The issues discussed in this section are by no means new topics in philosophy and sociology. 
Prenatal tests have been available for many years, and on that subject Harbers and Popkema discussed the political character of prenatal technology \cite{Harbers_2005}. 
These tests have a clear intentionality to detect children having a disorder and, in designing the test, the possible outcomes are being ordered into categories \cite{Stephenson_2017}. 
By providing a non-invasive technique with a high predictive value, the social and political effects that were already present are now of growing concern, especially because NIPT has been available to all pregnant women in the Netherlands since April 1, 2017 \cite{niptconsortium_2017a}. 
Harbers and Popkema argue that government regulation is only able to lead the use of technological advancements along the best possible ways. 
However \newline 

\hfill\begin{minipage}{\dimexpr\textwidth-1cm}
‘decisions [...] were made earlier and elsewhere – at the drawing table in the laboratory where the test was designed and in clinical settings where the test was regulated and further developed for practical implementation. 
In conjunction with the act of delegating competencies to a technological artifact like the serum-screening test during these early stages of design and development, a normative position has been smuggled into clinical practice.
This “incorporated normativity” appears to have a substantial impact on the room for choice and the action taken by relevant actors later in the process.
Pregnant couples, for example, get saddled with questions and responsibilities which they did not ask for and, more problematically, which they have trouble coping with.  
Meanwhile, politicians are held accountable for situations they did not create and can only marginally regulate.’ \cite{Harbers_2005}[p.231].
\end{minipage} \newline \newline

\noindent Where Harbers and Popkema were discussing the effects of serum-screening tests using plasma protein markers, the citation can be adopted without change for use in a discussion about NIPT, where the predictive value is much higher and screening can be extended to aberrations beyond just trisomies 13, 18 and 21 \cite{Bianchi_2014,Amorim_Costa_2017}. 
Therefore, Harbers and Popkema’s conclusion also holds for NIPT: \newline 

\hfill\begin{minipage}{\dimexpr\textwidth-1cm}
	‘[...] a simple piece of prenatal technology like a blood test is not a neutral, strictly technical, scientific affair. On the contrary, technologies like these deeply encroach upon the pregnancy and everything attached to it. First of all, the triple test is inextricably connected with a medical program focused on preventing defective lives. [...] Isn’t technology implicitly facilitating a society with which it is no longer self-evident that one can give birth to and care for a child with a congenital handicap? [...] this “therapy” [abortion] is enclosed in the diagnoses of serum-screening right from the outset.’ \cite{Harbers_2005}[p.238]
\end{minipage} \newline \newline

\noindent Harbers and Popkema continue by discussing how the large scale on which serum-screening is made available is creating a ‘“network of prenatal care” from which nobody can ultimately escape’[p.238]. 
Not participating in this network is a choice, just as much as participating is a choice. 
This means that the mere availability of the test already changes how people experience pregnancy. 
Not knowing if your child is a boy or a girl, or if it has Down syndrome or not, is a choice. 
In other words, choosing not to do NIPT is different from not having the choice. Some parents of a child with a genetic condition even refrain from future reproduction to avoid having to make a choice regarding prenatal testing \cite{Kelly_2009}. 
With the availability of NIPT, saying ‘no’ to the test is getting harder because the test has a high positive predictive value regarding the chromosomal status of the fetus and is without risk of induced miscarriage. 
This leads to what Harbers and Popkema call “anticipated decision regret” in which prospective parents try to avoid the future situation of “what if I had known before”’\cite{Harbers_2005}[p.238]. 

That these are not just arm-chair philosophical ponderings is shown by the situation in Denmark. 
After Denmark adopted a nation-wide prenatal screening program in 2004, the number of children born with Down syndrome decreased by 50\% \cite{Ekelund_2008,Lou_2018}. 
A 2015 newspaper article reported that 98\% of pregnancies carrying a fetus with Down syndrome were terminated \cite{Kastberg_2015}. 
This means that even before the introduction of NIPT, while using less reliable non-invasive screening methods in combination with invasive tests, almost all Danish women carrying a trisomy 21 child chose to have an abortion. 
One study showed that after NIPT became available, less invasive prenatal testing was done, but also that women who previously would not have had an invasive test now opted for NIPT \cite{Bjerregaard_2017}. 
However, many other women still opted for invasive testing because they wanted maximum information, including information about (sub)chromosomal aberrations other than the aneuploidies offered in NIPT \cite{Lou_2018}. 
A recent study in France highlighted the role of aversion of risk and ambiguity in the decision for choosing NIPT or a more informative invasive test \cite{Seror_2019}.

Now that NIPT has followed the serum-screening test in what Kater-Kuipers et al. call ‘routinization’ of the test \cite{Kater_Kuipers_2018} and is implemented worldwide \cite{Allyse_2015}, its possible negative effects should be considered. 
It has been argued that questions regarding the availability of NIPT are not about justice in healthcare, but about social justice \cite{Rolfes_2016}. 
The availability of prenatal testing without reimbursement for all creates social injustice, especially in low- and middle-income countries. 
To prevent social injustice, reimbursement of NIPT should be based on income rather than the a priori risk of having an affected child \cite{Rolfes_2016}. 
Others warn that NIPT should not be seen as a replacement of ultrasound testing during the first-trimester because ultrasound still is the most accurate method to detect various abnormalities \cite{Amorim_Costa_2017}.

It is possible that the ‘Danish scenario’, which they had already created before the availability of NIPT, will now take place in more countries. 
Even so, doom scenarios regarding tolerance towards having children with a congenital disease seem premature. 
At the moment in Denmark, the tolerance towards conceiving a child with Down syndrome does not seem to be affected and although the majority of people in Denmark find it a good thing that fewer children with Down syndrome are born, they also feel that there must be a place in society for children and adults with trisomy 21 \cite{Joachim_2017}. 
The question remains if these findings can be extrapolated to other congenital abnormalities that can now, or in the future, be detected by NIPT.

How then should we relate ourselves to NIPT? The choices opened up by the availability of this test were, largely, already opened up by other tests, most notably screening based on plasma protein markers and invasive prenatal DNA testing. 
If we take this background into account, the change made by introducing NIPT is that a highly reliable non-invasive test is now available for chromosomal aneuploidies, lowering the threshold to having a prenatal test, with a reduced number of follow-up tests necessary because of low false-positive rates. 
On a positive note, providing NIPT reduces the need for invasive tests and its associated risk of miscarriage. However, this only holds true if the high predictive value is maintained. 
As discussed earlier, the predictive value of NIPT is higher when the test is applied in a population where the frequency of a certain chromosomal aberration is higher. 
Now that NIPT has been made available for all pregnant women in the Netherlands, and not solely to women with a high risk of carrying a fetus with a trisomy, the percentage of false positive results will increase – since the average \textsl{a priori} chance of having a trisomy will decrease – possibly leading to more invasive procedures with a negative result.

Currently NIPT is becoming available in an increasing number of countries and regions. 
Unfortunately, in some regions there is insufficient access to professionals who can give adequate counselling \cite{Allyse_2015}. 
This might lead to children being aborted who don’t have a chromosomal abnormality because no follow-up is performed after a positive test. 
In addition, there is a cultural factor. For instance, in the People’s Republic of China, research has shown that 83\% of people receiving the result that there is a 5\% chance of their child having a “handicap” would have an abortion \cite{Allyse_2015}. 
On the other hand, one could argue that, since NIPT has a higher predictive value than serum-screening tests, more tests will show a very high or very low posterior risk, preventing such ‘middle’ outcomes. 
However, when expanding the test to detect less prevalent (sub)chromosomal abnormalities, this advantage will decrease. The reason for this is that for each chromosomal abnormality a separate prediction has to be made, each with a possibility of a false positive result \cite{Chitty_2018}. 
Obviously, by including detection of other aberrations in the analysis, more will be detected.

In our tool design (chapters 7 and 8) we opted to focus on whole chromosome aberrations only and to calculate a personalized risk instead of just providing a positive or negative result for trisomies 13, 18 and 21. 
It is up to politicians and healthcare providers to determine which aberrations should be tested for during pregnancy in the absence of a specific indication. 
What message does the wide-spread availability of prenatal testing for a wide array of (sub)chromosomal abnormalities, or the three most common abnormalities, without risk of miscarriage, give to the people? 
Are we just providing information that expecting parents can use to make informed decisions? 
Or should we see the intentionality of the test to detect abnormalities, in combination with government approval to offer the test to all pregnant women, as an ‘unwanted’ label on children having one of the syndromes present in the test? 
It has been argued that the health care system characteristics affect the uptake of people having a prenatal test \cite{Crombag_2016}. 
This shows that unintended powers are at work that inhibit autonomous decision-making, and these powers have become stronger with the introduction of NIPT, given its non-risk nature. 

I will not conclude this section with a clear opinion on whether or not the introduction of NIPT is a good thing or not, or if we should offer it or not, but rather say that more public discussion is needed. 
In particular, as health care providers and developers of techniques to detect chromosomal aberrations prenatally, we should enter this public debate and give clear information on what the tests entail and how they should be interpreted. 
Hopefully, with this section I have added a few points to consider in this debate. 

\section{The potential patient}
Genetic testing on hereditary diseases (e.g. using Sanger sequencing, targeted gene panel testing or whole exome sequencing (WES)) can lead to the discovery of pathogenic variants for which it is known that carriers have an increased risk of developing a disease. 
When this testing is used diagnostically, i.e. in individuals with a disease phenotype which might be genetic, finding such a variant can (partly) explain why they have developed the disease. 
However, when used in individuals who have not developed this disease, this genetic diagnosis might feel like Damocles’ sword hanging above their heads, waiting to fall. 
A famous example of such an individual is Angelina Jolie, in whom genetic testing was performed because of her family history of breast cancer. 
After being diagnosed as a carrier of a pathogenic \textsl{BRCA1} variant, which meant that she had a high risk of developing cancer herself, Jolie underwent a preventive bilateral mastectomy in 2013 \cite{Nabi_2017}. 
Two years later she had her ovaries and fallopian tubes removed \cite{Vicari_2017}. 
Although she has lost many family members, including her mother and grandmother to \textsl{BRCA1}-related cancer, a point can be made that the discovery of her carriership of the variant led her to become a pre-symptomatic patient. 
Others may argue that she already was. 
Verbeek makes the observation about the breast cancer genetic test that ‘[s]uch tests, which can predict the probability that people will develop this form of cancer, transform healthy people into potential patients and translate a congenital defect into a preventable defect [...]’\cite{Verbeek_2011}[p.57]. 
Angelina Jolie – and many other women with her – was presented with a choice that she probably wouldn’t have had without the presence of the genetic test. 

The test used to detect Jolie’s \textsl{BRCA1} variant was explicitly designed to detect the nucleotide sequence of the gene, but the availability of the test had other consequences, which Verbeek calls ‘implicit forms of mediation’[p.83]. 
Frederick Sanger did not intend to facilitate decision making on having a mastectomy or not. 
However, his work did just that. He may not have been able to foresee these consequences in 1975, but we certainly can now, having seen the power of genetic techniques in the past decades, particularly when designing gene panels within a clinical diagnostic setting. 
Furthermore, now that the gene-by-gene analysis paradigm within DNA diagnostics has been replaced by the sequencing of entire gene panels, whole exomes and even whole genomes, there is an increased chance of encountering pathogenic variants in genes that have no relation to the original reason for ordering that genetic testing. 
Such variants are called secondary findings, and the American College of Medical Genetics and Genomics (ACMG) has created a list of 59 genes for which they recommend reporting these secondary findings because they confer a high risk of disease and action is possible to prevent the disease or detect it in an early stage \cite{Kalia_2016}. 
As discussed in chapter 4, in this way the diagnostic procedure has expanded to a screening procedure. 
When clinicians adhere to the ACMG recommendations, the availability and use of a genetic test can lead to people obtaining knowledge about their having a risk of developing a specific disease. 
This is information they would not have had without having done the test. 
In other words, people carrying the same \textsl{BRCA1} variant as Angelina Jolie will now face the same decisions she had to make.

Although I appreciate the aim of preventing disease, the earlier-mentioned ‘potential patients’ will be diagnosed as ‘not-yet-ill’. 
When reporting variants and associated risk of developing a disease, whether as secondary findings or as the result of population screening, this is even more the case than it was for Angelina Jolie, who had a legitimate reason to be tested for BRCA1 variants and already felt at risk. 
To prevent overtreatment, when adhering to recommendations for opportunistic screening for certain genetic variants, care should be taken that the focus is on genes for which cancer risks and benefits of preventive interventions are clear. 
A further reason for this is that penetrance estimates are often based on individuals with a family history of a specific disease, and these estimates do not necessarily hold for the general population \cite{Wentzensen_2018}.

By stating that a person is predisposed to develop a disease and labelling them as an ‘asymptomatic carrier’, genetics seems to pull the definition of being affected by a disease earlier in time. 
Previously a person would be affected by a disease, for instance cancer, if the disease had developed, i.e. cancerous cells were present in the body. 
With genetic testing, an individual might be already labeled as being predisposed to cancer without having a single cancerous cell. 
Now a person with an elevated risk of developing cancerous cells is considered to be affected, even if it turns out that he or she never develops cancer. In some ways, the concept of ‘having’ a disease is being shifted away from the individual level and is now being considered on a population level. 
For instance, if 80\% of people with a specific DNA variant will develop a specific disease during their lifetime, we will call an individual carrier of this variant ‘affected’ but asymptomatic. 
In contrast, the detection of somatic variants (i.e. tumor analysis) only gives information about the current situation. 
A condition that is present at the moment of testing is diagnosed and, even though that person may still feel healthy, the disease is already having a physical effect. 
Similarly, testing for genetic variants in individuals with a congenital abnormality provides information regarding a current situation although, depending on which variant is detected, elucidating the current condition may entail future risks of development of other conditions. 

Verbeek connects the power of technology with the theory of ‘disciplinary power’ that Michel Foucault puts forward in his book Discipline and Punish \cite{Foucault_1978}. 
Here Foucault states that many relations of power exist in human society. 
The people in power determine ‘The Norm’, and in the hospital individuals are examined and classified [p.192].
The geneticist has the power to brand an otherwise healthy person as an asymptomatic patient. 
The choice to disclose a detected DNA variant is the choice to brand somebody as a patient, with sometimes dramatic consequences \cite{Fulda_2006}. 
This power is ‘at work’ through the sequencing technique, data analysis algorithms and variant interpretation tools \cite{Verbeek_2011}[p.70]. 

For germline variants associated to hereditary disease, this branding can be seen as making explicit a risk profile that was already present. 
But being at risk is different than being labelled as being at risk, especially if you consider life insurance and the possibility of genetic discrimination \cite{Dalpe_2017,erfelijkheid.nl_nd}. 
For somatic variants there is discrimination in a positive context. Here, genetics adds to precision medicine. For example, knowledge regarding the presence of somatic variations in leukemia bone marrow cells can help predict the effectiveness of certain treatments \cite{Motyckova_2010,Marum_2016,Winters_2017,Baccarani_2013}. 
Therefore, based on which somatic variants are present, it can be determined who is eligible to receive a specific treatment and who is not.

In addition, particularly for germline variants, knowing one’s DNA profile changes the way we relate to ourselves, to other people or to a (possible) pregnancy.
This psychological impact can even extend to family members who may consider themselves as possible carriers even though they have not been tested themselves. 
There is a clear intentionality in sequencing genes with a known disease-association. 
The goal is to find genetic defects that can explain a known condition or disease risk or predict if we, or our children, have an increased risk of developing a certain disease. 
Developing a genetic test or an analysis method is therefore not without moral responsibility. We should keep in mind that the availability of these tests has an impact on people’s lives. 
We, as designers of such tests, have created new choices, transforming people into potential patients. 
Not having a genetic test is also a choice, and having to choose to do a test or not can already be impact enough. 

\section{Revisiting existing data} 
With the availability of ever larger numbers of genetic variants due to the introduction of next-generation sequencing, more and more variants get reclassified over time \cite{Shah_2018}. 
This has led to the discussion of reanalysis, reinterpretation, reclassification and recontacting \cite{Carrieri_2018}. 
It has been suggested that reinterpretation is necessary to achieve the highest standard of care \cite{Chisholm_2017}. 
In this context reinterpretation means that DNA variants that have been previously detected are looked at again and reclassified if new evidence suggests they should now be classified differently than they were in the initial report (e.g. from VUS to Pathogenic). 
If this occurs, the patient with this variant can be recontacted. 
Reanalysis takes a further step back. Here, the existing raw data is reanalyzed, for instance using an updated read alignment tool or by the application of CNV calling tools to NGS data that was previously only analyzed for SNVs and Indels. 
In chapter 4 we showed that in 0.6\% of the patients analyzed using the familial cancer gene panel, a pathogenic or likely pathogenic CNV was found in a gene that was associated to the cancer phenotype that led to genetic testing. 
In our study the patient data was anonymized, but this means that, if only SNV and indel analysis was performed in the initial diagnostics, new diagnoses would have been made in a small portion of the patients after data reanalysis for CNVs. 

Such reanalysis has an impact on total diagnostic yield, but reanalysis may also be extended to secondary findings. 
We can ask ourselves what the consequence would be if we would adhere to ACMG recommendations on reporting of secondary findings. 
In their 2017 report the ACMG recommendations added four genes to the 2013 ‘minimal list’ \cite{Kalia_2016}. 
Should these genes be reanalyzed using an in silico gene panel for patients for whom WES was performed previously? 

Currently, in the Netherlands, (periodic and/or active) reanalysis and reinterpretation of data or variants is not routinely performed in most laboratories, and variants are only reinterpreted upon an external trigger, i.e. on request or if a variant is found again in another patient \cite{El_Mecky_2019}.
Reanalysis of diagnostic data is mostly performed in research projects, but may still result in a diagnosis for a patient \cite{RadboudUMC_2018,SolveRD_nd}. 

Because, the relationship between the genetics diagnostics center and the patient is extended via reanalysis of data and reinterpretation of variants, and recontact could take place many years after the initial test, procedures surrounding informed consent should be discussed. 
To prevent patients having to make a choice regarding what they want to know for an indeterminate amount of time, a shift towards dynamic consent \cite{Kaye_2015} may be necessary.

\sectionmark{Is genomic information your family's?} % extra to get the top mark okay..
\section[Does your genomic information belong to your family? ]{Does information about your genome belong to your family? }
\sectionmark{Is genomic information your family's?}
In the following paragraph I will digress from my discussion of genetic analysis to question if you are the sole owner of your own genetic information, or whether we should consider ownership to be shared between family members. 
This question is becoming increasingly relevant given the ever-growing amounts of genetic data produced, including complete genomes. 

An engaging presentation on personal genomics by Dawn Barry, then vice president of life science and applied markets at Illumina, at the 2016 TedXSanDiego is titled “There is nothing more personal than your genome” \cite{Barry_2016}. 
In her presentation she reveals that both of her parents died of cancer and did not respond to chemotherapy. 
After having her own genome sequenced she learned that she too “was unlikely to respond to the typical course of cancer chemotherapy”. 
She then goes on to say that this information might have helped her parents by allowing them to choose a different therapy, or no therapy at all. 
Next to her interesting commentary on how knowing your genome can help manage your life and choices, the connection between the title of the talk and the relation between Barry, her children and her parents is particularly interesting. 
How personal is your genome? 
Several times during the presentation Barry mentions that she is 50\% her mum and 50\% her dad. 
What if her parents had done genomic testing themselves? They then would have known this very personal information. 
However, they would also have known that there was a risk that their daughter too would not respond to typical cancer chemotherapy. 
After all, she is half mum and half dad. 
It seems, then, that genetic data is not personal at all, but rather interpersonal between generations. 

Who then should have data ownership? 
If, as is the case in the example above, a genetic finding is clinically relevant and carries possible clinical relevance for family members, should the person whose DNA is tested be the only one who can say what information will be shared? 
Sijmons, van Langen and Sijmons state that “in a way, an individual’s genetic diagnosis is also a family diagnosis” \cite{Sijmons_2011}. 
Within the framework of clinical genetics for hereditary diseases, this question is even more pressing, because here, unless it is a secondary finding, the genetic information is sought for clinical reasons. 
Moreover, in many cases, the only reason that a person gets referred to a genetics diagnostics center is because various family members have developed a specific disease, as was the case for Angelina Jolie. 
The 1998 ASHG guidelines state that patient confidentiality should come first. 
Confidentiality may only be breached in exceptional cases where “attempts to encourage disclosure on the part of the patient have failed; the harm is highly likely to occur and is serious, imminent, and foreseeable; the at-risk relative(s) is identifiable; and the disease is preventable, treatable, or medically accepted standards indicate that early monitoring will reduce the genetic risk” \cite{Knoppers_1998}. 
However, these guidelines were created before the first human genome was sequenced. Now, with the widespread availability of genome-wide sequencing, these guidelines need to be adapted to the new reality \cite{Wolf_2015}.

In a patient survey performed in the UK, participants indicated that, although they perceived their condition as personal, they considered their genetic information to be familial \cite{Dheensa_2016}. 
Some participants even considered it a duty to contact at-risk relatives. 
Others felt that the tested person had rights over the information that does not extend to family members. 
“The reason for this intuition was that the result was generated by doing a test on her sister’s body, with her consent and cooperation, and was contained within her blood” \cite{Dheensa_2016}[p.176]. 
Still, in general, participants felt that the possible harm for family members, which did not need to be imminent, was more important than patient confidentiality when considering the question of whether to disclose relevant genetic information to family members or not. 
Based on these findings Dheensa, Fenwick and Lucassen suggest that health practitioners should change their default position towards sharing clinically relevant genetic information with family members and that they should be enabled to share information if it is felt necessary, even if a patient refuses to do so. 
They also recommend that the practitioner discusses with the patient before the test that certain findings will be considered familial rather than personal. 
These findings are in agreement with those of Heaton and Chico, who concluded from another UK-based survey that most respondents said that relevant genetic information on serious and/or preventable conditions should be shared with relatives, even against the wishes of the tested person \cite{Heaton_2015}. 
Dove et al. seek the solution in a relational autonomy where, at the start of a clinical relationship, responsibilities are negotiated between clinician and patient \cite{Dove_2017}. 
One of these responsibilities entails the respect for preferences of third parties, including family members. 
In this way autonomy is not taken away from the individual, but the preferences of and duties to family members are taken into account.

Now, in the genomic era, it is often the case that many genes are analyzed and variants identified that are not initially sought after and have no relation to the reason why a person was tested. 
In other words, it is not the case that the occurrence of a severe, possibly preventable or treatable, disease in the family was the reason to be tested and the disease for which it is discovered that the patient has an elevated risk has nothing to do with the occurrence of this disease in the family. 
However, parents, siblings or children have a high risk of carrying this same variant. If knowledge of carrying a specific genetic variant leads to preventable harm, it is my opinion that sharing of secondary findings should be treated the same as sharing of sought-after variants. 
I believe that sharing polygenic risk scores with family members is less useful, since these are based on an interplay between different variants. 
This is because, with an increasing number of variants adding to the score, the likelihood of a family member carrying the same set of variants decreases. 
A polygenic risk score generally will thus not be very informative regarding risks of family members.

As a side note I want to initiate another discussion. 
Should people be allowed to make their own genome public without consent of their family? 
Sharing your genome with the general public is already happening and can have consequences for your privacy \cite{Smith_2017b}. 
Since, as discussed above, half of your genome is shared with your first-degree relatives, you’re also making half their genome public. 
This information can be used for identification of relatives \cite{Kim_2018,Erlich_2018}. 
If, for instance, you are a carrier of a pathogenic \textsl{BRCA1} variant, your sister has a 50\% percent chance of carrying the same variant. 
Currently such information is not allowed to be used by, for instance, insurance companies \cite{Bin_2018}. 
However, if such information is accessible in the public domain, we open the door for genetic discrimination. 
If an individual wants to give up their own privacy, that is their right. 
But we should not be allowed to give up the privacy of our family members to the general public. For this reason I would strongly discourage making one’s genetic information public. 

\section{Moralizing introduced methods and algorithms}
Now that we have discussed the many ways in which the methods and algorithms introduced in this thesis mediate human moral decisions and actions, the time has come for a verdict. 
In my opinion, even though working in genetics is like walking through a mine field of moral issues, our workflows and tools have been designed with care and they mitigate moral issues rather than increase them. 
The gene panels introduced and used in chapters 2, 3 and 4 are targeted gene panels, rather than WES or whole genome sequencing (WGS), meaning that there is less chance of detecting secondary findings, while maximizing the possibility of finding a diagnosis by including all known relevant genes.
Moreover, by enabling detection of (single exon) CNVs in the same gene panels, the same data can be used to further increase the diagnostic yield with only a very limited risk of secondary findings. 
As discussed in chapter 4, for familial cancer we recommend only analyzing genes with an established relation to the cancer type that led to referral because, for now, there is insufficient evidence of what the penetrance of pathogenic variants is in the absence of a family history. 
Furthermore, when performing WES or WGS, I would recommend only offering opportunistic screening for secondary findings in genes for which the risk of developing a disease, again in the absence of a family history, is known if a pathogenic variant would be detected. 
To prevent detection of anticipatable incidental findings within a diagnostic setting, I recommend using virtual subpanels and only analysing those variants that fall in genes or regions with known association to the condition that led to genetic testing.

Because almost all gene-panel findings are sought-after, people with a pathogenic variant may already feel that they are ‘potential patients’ prior to testing, while family members not carrying the variant may be relieved. 
Yes, the intention is to detect pathogenic variants and label a person as being at increased risk, but the intention is also to identify family members whose risk is not increased above population level. 
In other words, with the use of our diagnostic gene panels we are creating fewer patients rather than more, while those patients who are at risk will have more options to try and prevent development of disease. 

Because the TLA multiplex panel for translocation detection in acute leukemias we described in chapter 5 is not ready for diagnostic use yet, it will not mediate decision-making for now. 
However, if the method is improved and sufficient speed, sensitivity and specificity achieved, so that there is no need of further confirmation, I do not foresee any negative moral consequences. 
The intention is to detect translocations that can help with specific prognoses or treatment choices, and thus optimize treatment of leukemia patients. 
A patient could receive a worse prognosis than expected or not be eligible for a specific treatment because it has a low success rate in patients with a specific aberration. 
But, in my opinion, this is a positive point because treatment can now be personalized and patients can receive optimal treatment based upon expected utility of medicine and therapy. 

With the development of NIPT algorithms, as described in chapters 6, 7 and 8, we opted to only focus on detection of the three most common trisomies and provide a tool to interpret test results in the context of patient and test characteristics. 
With the exclusion of less common and subchromosomal aberrations from primary detection – even though it is possible to test for all chromosomal aneuploidies – we only test for those aberrations that can be detected with the highest positive predictive value. 
This means that there is a low number of false positive compared to true positive results and unnecessary invasive procedures are thereby prevented. 
In my opinion we should be careful to extend the number of aberrations for which a prediction is performed in order to prevent false positive results. 
For less prevalent aberrations (which have a low \textsl{a priori} chance of being present) a higher threshold could be set for detection so that, even though sensitivity may drop slightly, the highest specificity is guaranteed.
 
\section{Conclusion}
In this chapter I have explored issues created by enabling genetic testing and have tried to elucidate possible ethical consequences stemming from the availability of the tools and methods presented in this thesis. 
Many of the issues have to do with uncertainty: uncertainty in knowing what will be found, uncertainty regarding whether or not a disease will develop, uncertainty regarding possible overtreatment, uncertainty if there will be new findings in your genetic data at a later stage, uncertainty regarding changing classifications of variants, uncertainty regarding psychological burden for patients and uncertainty about the presence of a variant, just to name a few examples. 

With so many uncertainties, the expansion of the number of genetic tests performed, the part of the genome that is analyzed and the continuing improvement of variant detection and interpretation, it is important to pause for a moment with every development and reflect on possible (unintended) consequences that the availability of a method or a tool may have. 
Hopefully, this chapter provides at least some pause and moral reflection on the methods, algorithms and tools introduced in this thesis.
